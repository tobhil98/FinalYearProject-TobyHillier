The extensions for the project have not yet been defined, but there are a variety of different options. The current plan is to implement the features required in AirSim and then use them in an extension task. Listed below are some of the suggestions which have been discussed in the supervisor meetings. 
\\~\\
One option would be to try to use the simulator alongside another final year project to simulate the live feed from traffic cameras. There are several use cases for this. Firstly, it could detect dangerous driving and report it to the police. This could both make people drive more safely as they know they are being surveyed, as well as making it possible to stop dangerous driving early on. It could also be used to track dangerous traffic junctions to monitor and detect near-collisions. 
\\~\\
Another option could be to simulate an autonomous wheelchair which exists in the Imperial Robotics Lab. By modeling the wheelchair in AirSim we could try to create an autonomous system and then compare it to the behavior in real life.
\\~\\
A third example could be to try to train a machine learning model for an autonomous robot so that it could navigate around crowded places with cars and pedestrians \cite{ChaoQianwen2015Vifm}. 
\\~\\