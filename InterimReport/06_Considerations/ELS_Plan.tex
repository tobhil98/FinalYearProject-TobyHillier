\section{Ethical Considerations}
As this is just a simulation, there are not really any ethical concerns as such. 
\\~\\
If we however decide to use the knowledge we learn in our simulator on a physical system there are a few things worth thinking about. Ethical considerations for autonomous systems have been a discussion for a long time \cite{ArkinRonaldC2016EaAS, BorensteinJason2019SCaE}. There are a variety of different ethical concerns. To name just a few, will an autonomous system be responsible enough \cite{BorensteinJason2019SCaE}, who would be responsible for an accident involving a self-driving car \cite{Hevelke2015, EthicsIssue}, and how could autonomous vehicles impact peoples behavior \cite{moralComputers}. These are not ethical concerns for the project as is, but could become an issue once we decide what the plan for the future is (Section~\ref{FuturePlanning}). 

\section{Legal Considerations}
AirSim has an MIT license which means that we can use the simulator however we like \cite{MITLicense}. This is the same license that is used for StreetMap. In regards to the game engine, as we have chosen to use AirSim which uses Unreal Engine, we are free to distribute the simulator as long as we don't make a gross profit of more than \$1 million \cite{UE5}. 
\\~\\
It could also be worth considering what the UK rules for autonomous vehicles are\footnote{\url{https://assets.publishing.service.gov.uk/government/uploads/system/uploads/attachment_data/file/929352/innovation-is-great-connected-and-automated-vehicles-booklet.pdf}}, if in the future work we decide to introduce a physical system \cite{UKAutoRules, UKAutoRulesGov2}. 


\section{Safety Considerations}
In regards to the project itself, there are no safety issues as it is a simulator being worked on from home. 
\\~\\
As mentioned in the other two sections. If information from the simulator ever gets used in a physical product there are a few things worth considering. Firstly, it is important to remember these are only simulations. Training a machine learning model on the simulator will not accurately reflect the behavior in real life. Secondly, it is important when testing a physical system that it does not for example collide with someone. This could be prevented by driving slowly or not driving where other people or objects might be in the way.