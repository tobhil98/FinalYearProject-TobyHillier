%What is the problem


% Who cares if this project is done
 

% How does this project relate to other work

% What does it build on

This project will look at creating a simulator that is flexible and diverse to incorporate mixed traffic simulation. Mixed traffic simulation will be a simulation that can simulate the interaction between agents which are not all the same, for example, pedestrians, vehicles and other autonomous systems. As will be looked at in Section~\ref{analysisOfSimulators}, there are no easy to use, open-source simulators that allow users full flexibility to customise their environments, APIs and physics.

The project looks to create an open environment where users can simulate mixed traffic. The project is first and foremost aimed at researchers who need a platform to test their autonomous system. 

There are three alternatives for how to develop the simulator. The first option is to use an existing simulator as a tool and then create a mixed traffic simulation inside that environment. The next option is to modify and extend an existing simulator. The last option is to begin from scratch.   

After analysis of existing simulators, the project will look to extend AirSim. After trying to use Unreal Engine as the game engine, it was decided that Unity would allow for more flexibility.  

Carla would probably be the simulator that resembles this project the most. This is because it has a lot of features such as pedestrians and autonomous vehicles which all can be controlled through APIs. The main advantage of this project over Carla is the flexibility of the platform. To load new maps into Carla is challenging as the traffic rules have to be embedded in the map. Carla is also a stand-alone project rather than a plug-in like AirSim making adding new features harder. 