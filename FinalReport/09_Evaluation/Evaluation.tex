This chapter will look at what has been achieved and evaluate this against the requirement capture (Chapter~\ref{}). 


% Originally using UE but then moving to Unity


% When evaluating the simulator there will be some key features to look at:
% \\~\\
% The main one will be \textbf{usability}. It is important that it is easy to use the simulator. When evaluating the usability we will look at:
% \begin{itemize}
%     \item How to set up the simulator.
%     \item How easy it is to import maps.
%     \item How to add agents into the simulation.
%     \item How to configure these agents.
%     \item How to control these agents.
%     \item Are the APIs easy to use?
% \end{itemize}
% Documentation should be written to make it clearer in regards to the elements listed above. It could also be useful to ask someone else to try using these features to evaluate if the documentation is clear and intuitive. 
% \\~\\
% The simulator could also be evaluated on \textbf{feature richness}:
% \begin{itemize}
%     \item How much of the objectives (Section~\ref{Objectives}) have been implemented.
%     \item Are there any other needed features that have not been implemented.
%     \item Is there a variety of available agents, such as pedestrians and autonomous robots?
% \end{itemize}

% %\\~\\
% Finally, it is also worth evaluating the \textbf{performance}:
% \begin{itemize}
%     \item Memory usage
%     \item Frame rate
%     \item Responsiveness
% \end{itemize}
% Some simulators had a clear impact on performance when several agents were spawned at once. It is important to make sure that the simulator is still usable for a small number of entities. 

